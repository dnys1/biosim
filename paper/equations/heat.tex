\newpage
\section{Heat Transfer}
Equations related to the transfer of heat.
\begin{gather}
    Q = mc\Delta T \label{eq:heat_transfer} \\
    Q = m L
    \label{eq:latent_heat}
\end{gather}
%% ASSUMPTIONS
\subsection{Assumptions}
Assumptions for this section are:
\begin{itemize}
    \item Complete and instantaneous mixing of fluids
\end{itemize}
%% HEAT TRANSFER - WATER
\subsection{Water}
Heat transfer at the water surface. $J$ represents the total heat flux in W/m$^2$. Incoming radiation at the surface is divided into short-wave ($J_{sn}$) and long-wave ($J_{an}$) where short-wave radiation comes from the lights used to light the BioSim and long-wave radiation comes from the atmosphere itself. 

The total heat flux can be represented by the following equation \cite{chapra_surface_1997}:
\begin{equation}
    J=J_{sn}+J_{an}-(J_{br}+J_c+J_e)
\end{equation}
where
\begin{conditions*}
J & The total heat flux (W/m$^2$) \\
J_{sn} & Short-wave radiation flux (W/m$^2$) \\
J_{an} & Long-wave radiation flux (W/m$^2$) \\
J_{br} & Black-body radiation flux (W/m$^2$) \\
J_c & Convection-conduction radiation flux (W/m$^2$) \\
J_e & Evaporation-condensation radiation flux (W/m$^2$)
\end{conditions*}

\subsubsection{Short-wave radiation ($J_{sn}$)}
Short-wave radiation will be supplied to the water through overhead LED lights. Different lights will be available for growing and living spaces, each with different a light spectrum, energy requirement, and heat output.

To calculate short-wave radiation:
\begin{equation}
    J_{sn} = \frac{N_{\text{bulbs}}H_{\text{bulb}}}{A}
\end{equation}
where
\begin{conditions*}
J_{sn} & Short-wave radiation flux (W/m$^2$) \\
N_{\text{bulbs}} & The number of light bulbs \\
H_{\text{bulb}} & The heat per light bulb (W) \\
A & Area of lit space (m$^2$)
\end{conditions*}

\subsubsection{Long-wave radiation ($J_{an}$)}
Long-wave radiation is caused by the absorption, emission, and scattering of infrared radiation by the atmosphere. 

It can be calculated using the following equation \cite{chapra_surface_1997}:
\begin{equation}
    J_{an} = \sigma (T_{\text{air}}+273.15)^4 (A+0.031\sqrt{e_{\text{air}}/0.13332})(1-R_L)
\end{equation}
where
\begin{conditions*}
J_{an} & Long-wave atmospheric radiation flux (W/m$^2$) \\
\sigma & Stefan-Boltzmann constant ($5.67\times 10^{-8}$ W m$^{-2}$ K$^{-4}$) \\
T_{\text{air}} & Air temperature (\degree C) \\
A & A coefficient (0.5 to 0.7) \\
e_{\text{air}} & Air vapor pressure (kPa) \\
R_L & Reflection coefficient ($\approx 0.03$)
\end{conditions*}

\subsubsection{Black-body radiation ($J_{br}$)}
Black-body radiation is the long-wave radiation emitted from all objects. It is emitted from the water body and is responsible for the first term in the loss side of the total heat flux equation.

It can be calculated using the following equation \cite{chapra_surface_1997}:
\begin{equation}
    J_{br} = \epsilon \sigma (T_s + 273.15)^4
\end{equation}
where
\begin{conditions*}
J_{br} & Black-body radiation flux (W/m$^2$) \\
\epsilon & Emissivity of water ($\approx 0.97$) \\
\sigma & Stefan-Boltzmann constant ($5.67\times 10^{-8}$ W m$^{-2}$ K$^{-4}$) \\
T_s & Water surface temperature (\degree C)
\end{conditions*}

\subsubsection{Conduction/Convection ($J_c$)}
Conduction and convection both contribute to the total heat flux at the surface of the water. 

It can be calculated using the following equations \cite{chapra_surface_1997}:
\begin{gather}
    J_c = h_c (T_s - T_{\text{air}}) \\
    h_c = 10.45 - u + 10\sqrt{u}
\end{gather}
where
\begin{conditions*}
J_c & Conduction-convection heat flux (W/m$^2$) \\
h_c & Convective heat transfer coefficient (W m$^{-2}$ K$^{-1}$) \\
u & Wind velocity at the surface (m/s)
\end{conditions*}

\subsubsection{Evaporation ($J_e$)}
In order to calculate the heat transfer due to evaporation, the total mass of evaporated water is first calculated.
\begin{gather}
    m_e = (x_s - x) a_e \\
    a_e = \frac{25 + 19u}{3600}
\end{gather}
where
\begin{conditions*}
m_e & Mass of water evaporated (kg m$^{-2}$ s$^{-1}$) \\
x & Actual humidity ratio (kg/kg) \\
x_s & Saturation humidity ratio (kg/kg) \\
a_e & Evaporation coefficient (kg m$^{-2}$ s$^{-1}$) \\
u & Wind velocity at water surface (m/s)
\end{conditions*}
Then the total heat transfer can be calculated using the heat transfer formula \eqref{eq:latent_heat}:
\begin{equation}
    J_e = m_e L
\end{equation}
and by estimating the latent heat of evaporation of water \eqref{eq:latent_heat}.
\begin{conditions*}
J_e & Evaporative heat flux (J/m$^2$) \\
m_e & Mass flux of evaporated water (kg/m$^2$) \\L
\end{conditions*}

\subsubsection{Condensation ($J_d$)}
Condensation is a function of the temperature relative to the dew point. Thus, condensation only occurs if it is forced (i.e. through the use of a dehumidifier) or if the temperature drops below the dew point. The heat transfer due to dew forming can be calculated using the equation for heat transfer \eqref{eq:latent_heat}:
\begin{equation}
    J_d = m_d L
\end{equation}
where
\begin{conditions*}
J_d & Condensation heat flux (W/m$^2$) \\
m_d & Mass flux of dew deposited (kg m$^{-2}$ kg$^{-1}$) \\
L & Latent heat of evaporation (J/kg)
\end{conditions*}